"What is one of your favorite children's stories?"
Two children's books that I enjoyed reading as a child were the following. The third is one I recently learned to know and regret not having to read to my children.
Make Way for the Ducklings - Robert McClosky, 1941 https://www.youtube.com/watch?v=wxwVSUANofM 
From Wikipedia - Make Way for Ducklings is a children's picture book written and illustrated by Robert McCloskey. First published in 1941, the book tells the story of a pair of mallards who decide to raise their family on an island in the lagoon in Boston Public Garden, a park in the center of Boston.
Make Way for Ducklings won the 1942 Caldecott Medal for McCloskey's illustrations, executed in charcoal then lithographed on zinc plates. As of 2003, the book had sold over two million copies. The book's popularity led to the construction of a statue by Nancy Schon in the Public Garden of the mother duck and her eight ducklings, which is a popular destination for children and adults alike. 
I like maternal image of the mother duck who got her flock of ducklings to the park with the help of a friend and despite the busy streets. I enjoyed delightful illustrations as well.
Bright April - Marguerite De Angeli, 1946 https://en.wikipedia.org/wiki/Bright\_April 
From Wikipedia - Bright April is a 1946 children's story book written and illustrated by Marguerite de Angeli, who later won the 1950 Newbery Medal for excellence in American children's literature for The Door in the Wall. Bright April is a story about a young African-American girl named April who experiences racial prejudice; it is also the story of her bright personality and her tenth birthday and the surprise it brought. The story is set in the Germantown neighborhood of Philadelphia, Pennsylvania, and the scenery portrayed in the author's illustrations can be recognized even today.
Bright April was the first children's book to address the divisive issue of racial prejudice, a daring topic for a children's book of that time. Selected digital images of this book are available at the Marguerite de Angeli Collection. 
This was one of the first stories that I read as a child that included a little girl of color. I liked her and passed on to my children the code of DYB (Do your best.)
Enemy Pie by author -Derek Munson and illustrator -Tara Calahan King (https://www.youtube.com/watch?v=XraIfCpau20)
I learned to know this book in recent year and am sorry to not have it when my children were learning to deal with enemies. What a great way to respond to an enemy. Jeremy's dad had a great way of drawing Jeremy into the process of converting that enemy to a friend without Jeremy even knowing what was happening! If only more of us used this process in our personal lives and in our communities. May the pie makers increase.





