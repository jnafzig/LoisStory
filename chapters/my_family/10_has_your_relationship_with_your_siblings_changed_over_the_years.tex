\section{Has your relationship with your siblings changed over the years?}
Yes, I learned to know my oldest brothers and sister better when I returned from Honduras and entered the adult world of my community.
After I was married I felt like I had more in common with them.
Paul and Nancy, Dave and Jane went to the "home" church.
There was a point at which it was important for them to recognize that I was an adult and not one of the "little ones".
My sisters-in-law likely helped with that since I was close to adulthood when we met for the first time.

There was a time when I was in high school and college that it seemed to me that I did not register with my older siblings as a person.
One point at which I realized that was not completely true was the Sunday I left for Honduras.
Mom had invited everyone home for Sunday dinner and later that afternoon I would leave for the airport for the flight out of the country.
It was suggested that I be sent off with a time of singing.
So family members suggested songs for my sending off.
One of my brothers with whom I had a more distant relationship, Dave, suggested the song, "Under His Wings".
I remember getting choked up as I received that song as a sign of my brother's concern for me.

My relationships with my younger siblings has changed over the years.
As happened with my older siblings and me there were points at which I realized Rachel, Joe and Esther were taking on responsibility for themselves and others and were becoming adults.
Examples of this are when Rachel left for a year in Europe, Joe received permission from parents to take a motorcycle ride out west with several other young men and Esther enter VS program for a year or two.

There were times in my life that I worked to take my unique place in my family.
There was pressure to conform in some ways.
Some of my siblings made that easier than others.
All of them were supportive but I did not always feel that support.
Some of my sisters began to occasionally go out for a meal together to talk.
Those conversations were like good therapy when it came to processing our family system and the pain that we had experienced.
We could also affirm the positive experiences as well.
I was good for me to hear how others of my sisters had come through their growing up years and how they understood the family to be.

The fact that John and I moved our family to Indiana has impacted my relationships with my siblings.
Mary and I have reconnected after quite a few years of little connection.
The rest have become more distant connections.
The annual sister's gathering and the annual summer reunion have become quite important to me.
They are significant ways to keep in contact with my family.
Phone calls with sisters living in the east and regular sessions with Mary and Wishart provide a connection to family.
