\section{What is some of the best advice your father or mother ever gave you?}
Although my father ended his schooling after taking 8th grade twice and my mother when she turned 16 and had completed 11th grade, they both encouraged their children to get an education through high school.
My parents sent all of us to the local Mennonite elementary (Grades 1 through 8).
They were also willing to make the financial commitment of sending all of us to high school at the private Mennonite owned and operated school, Lancaster Mennonite School.
Until 1970, the year that I graduated, the school (LMS) was under the guidance of the Lancaster Mennonite Conference bishop board.
This is a group of ordained Mennonite church leaders.
In 1971 the head administrator was a professional educator.
He was not an ordained church leader.
While some of us shone brighter than others we all made it through high school.
Several of us have nurses training or masters degrees as well.
So valuing education was one piece of good advice they gave.


The second piece of advice was to do service.
Doing for other people in the name of Christ was important to my parents.
When they were getting married there was a strong missionary movement in their part of the Mennonite church.
Many people they knew were going to Africa or Central America.
They knew that they could not respond to that kind of call but decided that perhaps they could raise their children to serve others.
Most of my siblings and I have spent time serving in places of "need" here in the United States or internationally.
While I have come to understand that one can serve wherever one lives rather than going "out there", the opportunity to interact in new places and find our way in a new culture, was a valuable learning experience.
The advice to be of service to others was good.

% Jonathan: Why do you think you think your parents were so interested in having their children get more education than they did?
% My impression whas that Mennonite culture traditionally didn't value education (at least in the clergy)
% Was it related specifically to your mother not being able to continue her own education? 
% Did this differ from others in the community or was it part of a larger shift in Mennonite education
