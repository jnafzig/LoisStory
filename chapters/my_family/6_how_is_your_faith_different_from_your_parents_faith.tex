How is your faith different from your parents' faith?
I think that it would be fair to say that my parents had a very Biblical based faith. Their faith depended on the Bible being true and by true I suggest literally true. We were told and read stories that were Bible reality. We attended church regularly. The family gathered for daily Bible reading, singing and prayer. While I did not always understand what was read. I knew it was important to my parents. We did this to acknowledge the reality of God. God was a given in the world that I grew up in. I did become aware that the way some people lived was different than my family. I was also taught that the way we treated each other was important. Every person is of value. I will confess that I had a bias that people different that me were more interesting than people like me. 
As I grew to adulthood I had a hard time seeing the Bible as literally true. It soon became clear that the way one read the Bible made a difference in the way one lived. The Shema - Love the lord your God with all your heart, soul, and strength and your neighbor as yourself, became foundational for me. My parents would not have disagreed with the importance of that verse but they had made the commitment to the cultural life style aspects of faith as well. They were comfortable with keeping the two together. I was not. 
They believed that what one confessed should impact the way one lived. I also believed that. What that life looked like was different for me than it was for my parents. I felt drawn to living as an active peacemaker. I took mediation training. I came to seminary for a masters in peace studies, and took training Christian Peacemaker Teams. I'm not sure what my parents thought of these choices but they did not opening disagree with me or tell me I was wrong. 
My faith has continued to evolve and I live with more ambiguity and questions than ever. What does God mean? Who are we in the immensity (both large and small) of the universe? Answers to questions may not be all that important but I continue to find that relationships are the stuff of life and each day is made real by caring for the other. 
Charletta's question - Lois can you please elaborate on the distinctions between how you and your parents were impacted or how each one interpreted the Shema?
"The Shema - Love the lord your God with all your heart, soul, and strength and your neighbor as yourself, became foundational for me. My parents would not have disagreed with the importance of that verse but they had made the commitment to the cultural life style aspects of faith as well. They were comfortable with keeping the two together. I was not."
Reply to Charletta - It seemed to me that my parents thought about these verses in terms of their love of and obedience to God. I was finding that for me the expression of loving God is evident in how I love and care for my neighbor and myself. Let me know if you have more questions about that.
Tim's questions - Two follow up questions: Can you give an example of how your parents made choices based on their view of the Shema and its call to love and obedience to God and then share a story that exemplifies how you approach loving and caring for your neighbor. And then comparison and contrast.
Also: can you elaborate on this: "I will confess that I had a bias that people different than me were more interesting than people like me."
Reply from Lois - As I was waking up this morning the questions Charletta and Tim asked me were on my mind. Some thoughts about them came to me along with a song that I likely hadn't thought of or sung for a long time. I'll get to the thoughts later but here is the song that was going through my head. I found it in the hymnal from which my family sang on Sundays and in the daily family devotional time.

These words reflect some of the theology that was present in the community where I grew up. It reflects the first part of the Shema. There was an expectation that one should accept salvation through Jesus and live a life of piety, generosity, and goodness. My parents were very generous and offered hospitality. They opened their home to people different than them. Before I was old enough to remember my parents hosted a worker from Puerto Rico. He worked with my father on the farm and I believed lived in the house with my family. Later I remember them hosting large Sunday noon meals for other families from church. They hosted high schoolers who needed a place to stay over the weekend. Later after their children had left the home they frequently hosted families serving with Wycliffe as Bible translators. They were good people and I am not critical of them but rather I find that what motivates me to live a good life is different from what I heard in the community where I grew up. 
To begin let me again say that I'm less clear about God. Is God love, light, goodness, truth, and beauty? I'm uncomfortable making God responsible for all the events of this world let alone the details of my life. I have learned to love and care for myself in a way that frees me to care for others. I understand how there may be connections between living things in ways that could impact a shared survival or not. 
While my parents did not vote and were not politically involved in their community, I have, and have written letters to and spoken with political leaders. I've joined with groups calling for political action. I enjoyed the opportunities I have had to visit Capitol Hill and express my concerns and interests to leaders there. Some of the issues we talked about were death penalty, violence against women, environmental concerns, and others. 
While the mantra, "In any given situation, I am the one I can change" rings in my thinking I also know that the common good is important. We are linked in ways that can determine the survival of us all. I believe there are actions we each can take that can serve the common good.
Response to question about "I will confess that I had a bias that people different than me were more interesting than people like me." Perhaps this statement reflects my boredom with the ordinary. While I have lovely memories of the place I grew up, I enjoyed reading books that took me beyond the world in which I lived. I thoroughly enjoyed opportunities I had to travel away from Lancaster County. I'll name a few that I remember: South Carolina, Ontario, Italy, Switzerland, France, England, Honduras, Haiti, South Dakota, Mexico, Tanzania, Malawi, South Africa, Bolivia, Cuba, Arizona, Colorado, Washington, California. There were other states we passed through. I also was in each country of Central America. 
I have been curious about how much difference a group can tolerate. Can some kinds of difference put an end to the cohesion of a group of people?






