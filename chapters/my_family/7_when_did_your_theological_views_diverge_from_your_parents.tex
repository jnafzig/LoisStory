7 When did your theological views diverge from your parents? Describe the change and share related stories.
Perhaps I never felt comfortable within the closed community of Mennonites where I was born. I do not have memories of identifying with much of what happened in church on a Sunday morning. I do remember my fascination with my first Sunday school teacher, Emma Miller telling us stories using a flannel graph. It was visual and the people shown in the Bible stories were quite different looking than people in my everyday life. I heard the preachers teach of the importance of dressing in the prescribed Mennonite way. At some level I was processing these differences.
It was noted in my mother's diary under November 9, 1961 that, "Lois accepted the Lord this evening". On Saturday of that week she includes the note that "We went down to Landis' for coverings afterward." It seems that my mother understood the connection between those two statements but as a nine year old they did not seem to connect for me. I remember feelings of frustration that my mother now combed my hair and put it in a bun with a hairnet over it and hair pins that could jab one's head. Over top of that was placed a white covering held to my hair with straight pins. What had Brother Willis Kling preached about that night at the revival meetings at the Byerland church? I do not have clear memories of his sermon but I'm quite sure it was not an invitation to begin wearing plain clothes. However, in the community where I lived standing in response to a revival invitation was a public indication that one was ready to join church which meant dressing plain. I'm rather sure that my nine year old thinking was not ready to accept all of those actions at once. However I dressed as I was told to. A part of this perspective that did stay with me was the connection between what one said one believed and one's actions. I just did not personally make the connection to plain dress.
Another theological view that caused me problems was the hierarchical perspective on the roles of men and women. God was believed to be over everything, men were positioned beneath God and women were expected to take a role under men. Where you found men and where you found women made visible the distinct roles they had in the community where I grew up. Men were in the field driving tractors or on the road driving trucks. They most often worked with animals and harvested crops. Sunday morning it was a man who looked down from the pulpit and a man who lead the congregation in singing. During the week women could be found in the kitchen and in the garden. It was women who kept the house clean and laundry washed and dried. They most often packed the lunches and got the children ready for school and church. In school during the time I was in grades one to eight, I had eight different single women as teachers. Sunday morning you would find the women on the designated side of the auditorium since there was no mixed sitting of men and women. Women were expected to join the men in singing with their soprano and alto voices. They would likely be found teaching the children's Sunday school classes as well. 
One of the ways that I became aware of other perspectives regarding the roles of women was from reading books. Many alternative ways for women to live in the world could be found between the covers of a book. However, I need only sense the occasional tension between my parents to learn that the traditional role did not always fit naturally with their personalities. My mother enjoyed interacting with people and had a personality that could be useful in a leadership role. My father may have enjoyed people but took leadership roles reluctantly. These are my observations and are not based on specific things my parents told me. 
Thinking back I believe that I had a growing sense that I wanted independence from the female role expectations of the culture in which I grew up. However, I still remember the shock and surprise I felt when a male member of the youth group called me a feminist. I believe he intended to be derogatory and in that setting he was, but internally I wore it as a badge of courage. 
I found I was more likely to talk with my mother then my father about my different perspectives but one of the realities that helped me out was having older brothers and sisters who also found themselves differing with their parents on some issues. The way we dressed was one issue. In many ways they cleared the road and by the time I came along there seemed to less need to talk difficult issues over with my parents. 
While my theological perspectives were changing during high school years I was still figuring out what I thought for myself. My parents gave me permission to go on a five week trip to Europe and following that supported my plan to get a degree in education from Millersville State College. I was testing ideas and having experiences that I knew were different than my parents so this was the time when I was using my wings and learning to fly on my own. I'm not really sure how they felt about that but I was their eighth child so I doubt I kept them up at night very often.
Comment from Abby - That photo of you with the snowman is one of my favorites. I love the line about you wearing the label of feminist as an internal badge of courage.





