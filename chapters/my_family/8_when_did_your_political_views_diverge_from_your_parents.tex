8 When did your political views diverge from your parents? Describe the change and share related stories?
Growing up I knew very little about politics. Between the community where I lived and the rest of the world there was a boundary that let in small amounts of talk about elections, governors, taxes, presidents and public schools. I'm sure my parents paid taxes but I do not know how large the checks to the state and nation were. Did having a large family make a difference back then? I was aware of the fact that my sisters and brothers all went to a private parent run school where my parents paid tuition in addition to the taxes that help to pay for public schools. I thought that an indication of my parent's commitment to living the way they believed. 
I'm sure that my high school teachers provided windows through the boundary to the world outside. Knowing there were other ways to live and think about life resulted in my evaluating life the way I was brought up. I do not have memories of conversations with my parents about politics. I was more likely to have conversations with older brothers and perhaps sisters about these things. Many of my older siblings had moved out of the area even out of the state and country. Indirectly that provided different ways of seeing the world. My parents were always respectful of the USA. It would have been hard for them to be very critical of the government. They did not approve of war but that was part of what kept up the boundary between the community and the outside world. During high school and on into college I became aware of some of the sad realities of life beyond my small world. I learned about injustice and corruption. 
Living in Honduras I saw poverty and wealth in new ways.
I had thought poverty to be a result a of a person's poor choices. I now realized there is a systemic reality that contributes to poverty and the USA systems and structures contribute to keeping many people below the poverty line. Understanding that systems have impacted economic, race, religion, education and many other aspects of life was an important learning for me. I had a growing understanding that I had a voice and vote that could be used to help bring about change.
Using my vote is one way that I differed from my parents. They did not vote. I'm not sure but I think that the first time I voted was after I returned from Honduras. My political views was evolving and they would continue to evolve in the following years.
I'll say a bit more about what I understand my parent's perspective on politics to be. They understood the political system to be part of the earthly kingdom. They were committed to the heavenly kingdom of God. The Biblical teaching was a guide for how they lived. Matthew 5 to 7 were important directives for how one lives. The beatitudes were verses we memorized. The blessing for peacemakers was one of special importance. We were part of a community that opposed participation in war. However, at that time, church leaders did not tell the government they should not go to war. The government was the earthly kingdom. During WWI some Mennonite men chose to resist becoming soldiers. At that time there was no exemption for conscientious objectors to war. There are stories of suffering for the stand they took. I honestly do not know the date of the decision but before WWII the US government made it legal for a man with beliefs of conscientious objection to war to not join the military. He had the option of alternative service. There was a variety of jobs men could choose from. There were smoke jumpers fighting wild fires, caregivers in hospitals for the mentally ill, men serving as Guinea pigs for research and more. My father was given a farm deferment since he was needed to keep the farm operating and productive. It is interesting to note that the Mennonite leaders spoke to the government when there was an issue that concerned their lives and young people but were silent when other young people's lives were at stake.
In the years since the CO status became recognized by the government, there are more Mennonites who voice their concerns to the government. There is now an office run by the Mennonite Central Committee in Washington DC. Their goal is to witness to those making political decisions in the government. They invite partners in the international community to also come and bring their concerns and prospective to law makers. These perspectives are that of peacemakers seeking justice, equity, opportunity, health and education for all people. I'm not really sure what my parents thought about the choices I made to begin using my vote, get training to conduct mediation, study for a peace studies degree, and joining and serving with Christian Peacemakers Teams. I like to think that if they had been in the generation that I am and had the opportunities I have had they too would have made choices that were clearly that of peacemakers.
Tim's question - Mom, do you remember any conversations with your parents or early conversations with your siblings around these changes you describe?
Mom's reply - Unfortunately I do not have clear memories of such conversation.






